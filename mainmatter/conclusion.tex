\chapter{Concluding remarks}

This thesis presents a comprehensive study on multiphysics topology optimization in nanophotonics, focusing on developing novel design
methods for applications that rely on coupled physical effects, such as thermo-optical, optomechanical, and electro-optical interactions.

First, the field of nanophotonics and the numerical solution of Maxwell's equations are introduced, laying the foundation for single-physics topology optimization. The framework is then extended to address multiphysics
problems, and a general multiphysics topology optimization formulation is presented.

Thermo-optical topology optimization is considered next, focusing on the design of low-loss phase shifters through the optimization of metallic heater layouts around optical waveguides. This represents the first coupled multiphysics topology optimization formulation
in the nanophotonics literature, demonstrating compact and efficient heater designs that reduce optical losses by
$\approx 33\%$ compared to state-of-the-art solutions.

Next, several optomechanical topology optimization problems are explored. First, a topology optimization framework based on the Maxwell stress tensor formalism is developed and used to optimize the design of particles and their surrounding environments, with the goal of enhancing optical forces on particles of arbitrary size and shape.
It is shown that tailoring both the particle geometry and its
environment can significantly enhance momentum exchange with an optical field, achieving force enhancements of up to
$\approx 13\times$ compared to a reference design. Additionally, integrated photonic cavities are designed to enable omnidirectional trapping of sub-wavelength particles without relying on SIBA effects. Theoretical analysis demonstrates that these cavities achieve a trapping stiffness an order of magnitude greater than that of conventional diffraction-limited optical tweezers. Finally, optomechanical membranes are designed, accounting for the interaction between their optical response and mechanical deformation. Including the
optical-force-induced deformation in the design process, results in a performance improvement of $\approx 3\times$ over designs that
neglect this effect, highlighting the importance of including multiphysics interactions in the design process.

Lastly, electro-optical topology optimization is addressed. An efficient formulation for nanolaser design is derived and applied to design two- and three-dimensional nanolasers. It is demonstrated that the use of
 a laser physics-based figure of merit, which accounts for factors such as extended gain media and steady-state diffusion, outperforms
  heuristic local-density-of-states-like metrics, yielding up to a $\approx 3\times$ improvement.

Overall, this research contributes to the design of novel nanophotonic devices that leverage multiphysics interactions, enabling enhanced functionality, improved performance, and new capabilities in nanophotonic systems.

\section{Future work}

This work opens several avenues for future research, including:

\begin{itemize}
    \item \textbf{Application to experimental setups:} While the framework was developed theoretically, it can be adapted to specific experimental configurations with minor, problem-dependent modifications.
    
    %\item \textbf{Nonlinear multiphysics extensions:} The current adjoint formulation (\secref{sec:topopt_theory}) handles cascaded nonlinear dependencies in coupled systems, where the couplings are nonlinear but physics system can still be solved linearly. Future work could extend this to fully nonlinear systems, where each governing equations depend on their respective state solution.
    
    \item \textbf{Simultaneous multiphysics effects:} Extending the framework to handle multiple coupled effects, such as thermo-electro-optical interactions or combined geometric deformation and photoelasticity (\secref{sec:mech_strongly_coupled}), could unlock advanced device functionalities.
    
    \item \textbf{Advanced figures of merit (FOMs):} More sophisticated FOMs could be developed to better capture relevant physical effects. As an example, in the design of optically manipulated particles (\secref{sec:engi}), one could target optical torque (\eqref{eq:torque}) or design the spatial distribution of forces for engineered particle motion.
    
    \item \textbf{Strongly coupled multiphysics systems:} Beyond the optomechanical membrane problem studied in~\secref{sec:mech_strongly_coupled}, other strongly coupled problems, like heat-induced refractive index changes from optical absorption (\secref{sec:thermo_strong_coupling}), could be investigated.
    
    %\item \textbf{Nanolaser pumping models:} The nanolaser topology optimization framework (\secref{sec:laser}) could be extended to include explicit models of optical or electrical pumping.
\end{itemize}