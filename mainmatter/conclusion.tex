\chapter{Concluding remarks}

This thesis presents a comprehensive study on multiphysics topology optimization in nanophotonics, focusing on the development of novel design methods for applications that rely on coupled physical effects, such as thermo-optical, optomechanical, and electro-optical interactions.

It begins by introducing the field of nanophotonics (\secref{sec:nanophotonics}) and the numerical solution of Maxwell's equations (\secref{sec:fem}), which form the foundation for topology optimization in the single physics picture. The framework is then extended to multiphysics problems (\secref{sec:coupled}), including a general multiphysics topology optimization formulation (\secref{sec:topopt_theory}).

Thermo-optical topology optimization is addressed next (\chapref{chap:eo}), where heat transfer is coupled to Maxwell's equations. Several coupling mechanisms are considered, with a focus on designing low-loss thermo-optical phase shifters (\secref{sec:TOPS}) by strategically placing metallic heaters around optical waveguides.

The thesis then explores optomechanical topology optimization (\chapref{chap:om}), including the design of particles and their environments for optical force manipulation (\secref{sec:engi}), photonic cavities for sub-wavelength particle trapping (\secref{sec:dip}), and membrane structures whose optical response is modulated by mechanical deformation (\secref{sec:mech_strongly_coupled}).

Finally, electro-optical topology optimization is investigated (\chapref{chap:eo}), with an efficient design formulation for nanolaser devices (\secref{sec:laser}). Here, the interplay between carrier dynamics and optical modes is shown to be critical for optimizing device performance. 

Overall, this research contributes to the design of novel nanophotonic devices that leverage multiphysics interactions, enabling enhanced functionality, improved performance, and new capabilities in nanophotonic systems.
\section{Future work}

This work opens several avenues for future research, including:

\begin{itemize}
    \item \textbf{Application to experimental setups:} While the framework was developed theoretically, it can be adapted to specific experimental configurations with minor, problem-dependent modifications.
    
    \item \textbf{Nonlinear multiphysics extensions:} The current adjoint formulation (\secref{sec:topopt_theory}) handles cascaded nonlinear dependencies in coupled systems, where the couplings are nonlinear but physics system can still be solved linearly. Future work could consider Future work could extend this to fully nonlinear systems, where each governing equations depend on their respective state solution.
    
    \item \textbf{Simultaneous multiphysics effects:} Extending the framework to handle multiple coupled effects, such as thermo-electro-optical interactions or combined geometric deformation and photoelasticity (\secref{sec:mech_strongly_coupled}), could unlock advanced device functionalities.
    
    \item \textbf{Advanced figures of merit (FOMs):} In some works more complex FOMs could be explored. As an example, in the optical force engineering problem (\secref{sec:engi}), one could target optical torque (\eqref{eq:torque}) or design the spatial distribution of forces for engineered particle motion.
    
    \item \textbf{Strongly coupled systems:} Beyond the optomechanical membrane case, other strongly coupled problems, like heat-induced refractive index changes from optical absorption (\secref{sec:thermo_strong_coupling}), could be investigated.
    
    \item \textbf{Nanolaser pumping models:} The nanolaser topology optimization framework (\secref{sec:laser}) could be extended to include explicit models of optical or electrical pumping.
\end{itemize}