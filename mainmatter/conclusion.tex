\chapter{Concluding remarks}

This thesis presents a comprehensive study on multiphysics topology optimization in nanophotonics, focusing on developing novel design
methods for applications that rely on coupled physical effects, such as thermo-optical, optomechanical, and electro-optical interactions.

We begin by introducing the field of nanophotonics (\secref{sec:nanophotonics}) and the numerical solution of Maxwell's equations
(\secref{sec:fem}), which form the foundation for single-physics topology optimization. We then extend the framework to multiphysics
problems (\secref{sec:coupled}) and present a general multiphysics topology optimization formulation (\secref{sec:topopt_theory}).

Next, we address thermo-optical topology optimization (\chapref{chap:eo}), focusing on designing low-loss phase shifters
 (\secref{sec:TOPS}) by optimizing metallic heater layouts around optical waveguides. We derive the first coupled multiphysics topology optimization formulation
in the nanophotonics literature and demonstrate compact, efficient conduction heater designs that reduce optical losses by
$\approx 33\%$ compared to state-of-the-art solutions.

We then explore optomechanical topology optimization (\chapref{chap:om}). First, we design particles and their environments for optical
force engineering (\secref{sec:engi}) by developing a topology optimization framework that targets optical forces on particles of
arbitrary sizes and shapes using the MST formalism. Our optimized designs show that tailoring both the particle geometry and its
environment can significantly enhance momentum exchange with an optical field, achieving force enhancements of up to
$\approx 13\times$ compared to a reference design. Next, we design integrated photonic cavities for sub-wavelength particle trapping
(\secref{sec:dip}) and theoretically demonstrate omnidirectional trapping without relying on SIBA effects, while achieving a trapping stiffness
an order of magnitude larger than that of diffraction-limited optical tweezers. Lastly, we optimize optomechanical membranes
(\secref{sec:mech_strongly_coupled}) whose optical response strongly couples to mechanical deformation. By accounting for
optical-force-induced deformation in the design, we achieve a performance improvement of $\approx 3\times$ over designs that
neglect this effect, highlighting the importance of including multiphysics interactions in the design process.

Finally, we investigate electro-optical topology optimization (\chapref{chap:eo}), where we derive an efficient formulation for nanolaser design
(\secref{sec:laser}). We apply this approach to two- and three-dimensional nanolaser design and show that using
 a laser physics-based FOM (accounting for factors such as extended gain media and steady-state diffusion) outperforms
  heuristic LDOS-like metrics, yielding up to a $\approx 3\times$ improvement.

Overall, this research contributes to the design of novel nanophotonic devices that leverage multiphysics interactions, enabling enhanced functionality, improved performance, and new capabilities in nanophotonic systems.

\section{Future work}

This work opens several avenues for future research, including:

\begin{itemize}
    \item \textbf{Application to experimental setups:} While the framework was developed theoretically, it can be adapted to specific experimental configurations with minor, problem-dependent modifications.
    
    \item \textbf{Nonlinear multiphysics extensions:} The current adjoint formulation (\secref{sec:topopt_theory}) handles cascaded nonlinear dependencies in coupled systems, where the couplings are nonlinear but physics system can still be solved linearly. Future work could extend this to fully nonlinear systems, where each governing equations depend on their respective state solution.
    
    \item \textbf{Simultaneous multiphysics effects:} Extending the framework to handle multiple coupled effects, such as thermo-electro-optical interactions or combined geometric deformation and photoelasticity (\secref{sec:mech_strongly_coupled}), could unlock advanced device functionalities.
    
    \item \textbf{Advanced figures of merit (FOMs):} More sophisticated FOMs could be developed to better capture relevant physical effects. As an example, in the optical force engineering problem (\secref{sec:engi}), one could target optical torque (\eqref{eq:torque}) or design the spatial distribution of forces for engineered particle motion.
    
    \item \textbf{Strongly coupled systems:} Beyond the optomechanical membrane problem studied in~\secref{sec:mech_strongly_coupled}, other strongly coupled problems, like heat-induced refractive index changes from optical absorption (\secref{sec:thermo_strong_coupling}), could be investigated.
    
    \item \textbf{Nanolaser pumping models:} The nanolaser topology optimization framework (\secref{sec:laser}) could be extended to include explicit models of optical or electrical pumping.
\end{itemize}