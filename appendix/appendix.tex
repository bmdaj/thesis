\section{Sensitivity analysis for weakly coupled multiphysics problems}\label{app:appendix1}
In this appendix, we derive the adjoint sensitivity expressions for a general weakly coupled multiphysics
 problem with $N$ physics. The system is written as $\mathbf{K}\mathbf{u} = \mathbf{f}$, where $\mathbf{K}$ is 
 the stiffness matrix of the coupled system, $\mathbf{u}$ is the solution vector, and $\mathbf{f}$ is the source or forcing term vector. For 
 simplicity, we assume real-valued problems, but this method can be extended to complex-valued cases, as shown
  in~\cite{ownpub0} for a simple two-physics ($N=2$) example. As described in~\secref{sec:topopt_theory}, 
  for $N$ physics with \textbf{weak nonlinear coupling} (\eqref{eq:multiphysics_weak_nonlinear}), the full 
  system is upper or lower triangular, as in
\begin{equation} \label{eq:app_multiphysics_weak}
    \begin{bmatrix}
        \mathbf{K}_1(\mathbf{u}_2, \cdots \mathbf{u}_N)    & \mathbf{C}_{12} (\mathbf{u}_2, \cdots \mathbf{u}_N)& \cdots & \mathbf{C}_{1N}(\mathbf{u}_2, \cdots \mathbf{u}_N) \\
        0 & \mathbf{K}_2 (\mathbf{u}_3, \cdots \mathbf{u}_N)   & \cdots & \mathbf{C}_{2N} (\mathbf{u}_3, \cdots \mathbf{u}_N)\\
        \vdots          & \vdots          & \ddots & \vdots          \\
        0& 0 & \cdots & \mathbf{K}_N
    \end{bmatrix}
    \begin{bmatrix}
        \mathbf{u}_1 \\
        \mathbf{u}_2 \\
        \vdots       \\
        \mathbf{u}_N
    \end{bmatrix}
    =
    \begin{bmatrix}
        \mathbf{f}_1\\
        \mathbf{f}_2\\
        \vdots       \\
        \mathbf{f}_N
    \end{bmatrix}\,,
\end{equation}
where we have chosen the upper diagonal version of the weakly coupled problem and the system of equations has been rewriten in terms of the stiffness ($\mathbf{K}_i$) and 
coupling matrices ($\mathbf{C}_{ij}$) so that the right-hand-side $\mathbf{f}=[\mathbf{f}_1, \cdots, \mathbf{f}_N]^\top$ does not depend
on the solution vector $\mathbf{u}=[\mathbf{u}_1, \cdots, \mathbf{u}_N]^\top$. This notation is equivalent to having $N$ equations of the form
\begin{align}
\left\{
    \begin{aligned}
        \quad \mathbf{K}_1(\mathbf{u}_2, \ldots, \mathbf{u}_N)\, \mathbf{u}_1  + \sum_{j=2}^N \mathbf{C}_{1j} (\mathbf{u}_2, \cdots \mathbf{u}_N)\, \mathbf{u}_j &= \mathbf{f}_1\,, \\
        \quad\mathbf{K}_2(\mathbf{u}_3, \ldots, \mathbf{u}_N)\, \mathbf{u}_2 + \sum_{j=3}^N \mathbf{C}_{2j} (\mathbf{u}_3, \cdots \mathbf{u}_N)\, \mathbf{u}_j &= \mathbf{f}_2\,, \\
        \quad\ldots \\
        \quad\mathbf{K}_{N-1} (\mathbf{u}_N) \, \mathbf{u}_{N-1} + \mathbf{C}_{N-1\, N}(\mathbf{u}_N)\, \mathbf{u}_N &= \mathbf{f}_{N-1} \,, \\
        \quad\mathbf{K}_N \, \mathbf{u}_N &= \mathbf{f}_N\,.
    \end{aligned}
\right.
\end{align}
In this case, we can solve the system of equations sequentially in a \textbf{seggregated} fashion,
 where problem $i=N$ is solved first, and the solution is used to solve the next problem ($i=N-1$), and so on, until the first problem ($i=1$) is solved
 (as in the sequence: $\mathbf{u}_N \to \mathbf{u}_{N-1} \to \cdots \to \mathbf{u}_1$).\\

To calculate the sensitivites using the adjoint method we start by rewriting the FOM ($\Psi$) by adding the residual
of the state equations, multiplied by a set of Lagrange multipliers ($\Lambda_i$)
\begin{equation}\label{eq:adj_init}
    \Psi^\prime =\Psi + \sum^N_{i=1} \Lambda_{i}^{\top}\left(\mathbf{K}_i \mathbf{u}_i -\mathbf{f}_i + \sum^N_{j=i+1} \mathbf{C}_{ij} \mathbf{u}_j \right)\,.
\end{equation}
Now we calculate the sensitivities by taking the derivative of the FOM with respect to the physical field ($\hat{\rho}$)
\begin{align}\label{eq:sens_init}
    \frac{\d \Psi^\prime}{\d \hat{\rho}} 
    = \frac{\partial \Psi}{\partial \hat{\rho}} 
    &+ \sum_{i=1}^N \mathcal{D}^{(i)}[\Psi]
    + \sum_{i=1}^N \Lambda_i^\top \Bigg\{\Bigg(
        \frac{\partial \mathbf{K}_i}{\partial \hat{\rho}} 
        + \mathcal{D}^{(i+1)}[\mathbf{K}_i]\Bigg) \mathbf{u}_i 
        + \mathbf{K}_i \mathcal{D}^{(i)}[\mathbf{u}_i] \nonumber \\
        - \frac{\partial \mathbf{f}_i}{\partial \hat{\rho}} 
        &+ \sum_{j=i+1}^{N} \Bigg[ \Bigg(
            \frac{\partial \mathbf{C}_{ij}}{\partial \hat{\rho}} + 
            \mathcal{D}^{(i+1)}[\mathbf{C}_{ij}] \Bigg)\, \mathbf{u}_j 
            + \mathbf{C}_{ij} \mathcal{D}^{(j)}[\mathbf{u}_j] 
        \Bigg]
    \Bigg\}. 
\end{align}
where we have defined $\mathcal{D}^{(i)}[a]$ as the cascaded differential operator acting on $a$ to compactly write the derivative
\begin{align}
    \hspace*{-0.775cm}\mathcal{D}^{(i)}[a] &= \sum^N_{i} \frac{\partial a}{\partial \mathbf{u}_i} \Bigg[ \frac{\partial \mathbf{u}_i}{\partial \hat{\rho}} + 
    \sum_{j=i+1}^{N} \frac{\partial \mathbf{u}_i}{\partial \mathbf{u}_j} \Bigg( \frac{\partial \mathbf{u}_j}{\partial \hat{\rho}} + 
    \sum_{k=j+1}^{N} \frac{\partial \mathbf{u}_j}{\partial \mathbf{u}_k} \Bigg( \frac{\partial \mathbf{u}_k}{\partial \hat{\rho}} + \cdots \Bigg) \Bigg) \Bigg] \,\,\,\, \forall i \in [1, N] \,.
\end{align}
The chain-rule-based derivative expression in \eqref{eq:sens_init} takes into account all the dependencies of the solutions in the weakly coupled problem (\eqref{eq:app_multiphysics_weak}).

Taking \autoref{eq:sens_init} and grouping the terms it is possible to rewrite the sensitivities as
\begin{align}
    \frac{\d \Psi^\prime}{\d \hat{\rho}}  &= \frac{\partial \Psi}{\partial \hat{\rho}} + \sum^N_{i=1} \Lambda_{i}^{\top} \left( \frac{\partial \mathbf{K}_i}{\partial \hat{\rho}} \mathbf{u}_i - \frac{\partial \mathbf{f}_i}{\partial \hat{\rho}} + \sum_{j=i+1}^{N} \frac{\partial \mathbf{C}_{ij}}{\partial \hat{\rho} } \mathbf{u}_j \right) + \\ &+ \frac{\partial \mathbf{u}_1}{\partial \hat{\rho}} \big( \mathcal{A}_1 \big) + \frac{\partial \mathbf{u}_2}{\partial \hat{\rho}} \left( \mathcal{A}_1 \frac{\partial \mathbf{u}_1}{\partial \mathbf{u}_2} + \mathcal{A}_2 \right) + \\
    &+ \frac{\partial \mathbf{u}_3}{\partial \hat{\rho}} \left[ \mathcal{A}_1 \left( \frac{\partial \mathbf{u}_1}{\partial \mathbf{u}_3} + \frac{\partial \mathbf{u}_2}{\partial \mathbf{u}_3}\frac{\partial \mathbf{u}_1}{\partial \mathbf{u}_2}\right) + \frac{\partial \mathbf{u}_2}{\partial \mathbf{u}_3} \mathcal{A}_2 + \mathcal{A}_3 \right] + \cdots
\end{align}
where $\mathcal{A}_i (\Lambda_i)$ are the adjoint equation terms
\begin{equation}
    \mathcal{A}_i = \frac{\partial \Psi}{\partial \mathbf{u}_i} + \Lambda_{i}^\top \mathbf{K}_i + \sum_{j=1}^{i-1} \Lambda_j^\top\left(
     \frac{\partial \mathbf{K}_j}{\partial \mathbf{u}_i}\mathbf{u}_j + \mathbf{C}_{ji} + \sum^N_{k=j+1} \frac{\partial \mathbf{C}_{jk}}{\partial \mathbf{u}_i} \mathbf{u}_k  \right) \quad \quad \forall i \in [1, N] \,,
\end{equation}
which can be set to zero ($\mathcal{A}_i=0$) by solving the adjoint problems. 
To solve the adjoint equations one needs to find the Lagrange multipliers $\Lambda_i$ as the solution to the adjoint equations
\begin{equation}\label{eq:app_adj_eqs}
     \mathbf{K}^\top_i \Lambda_i = -\left(\frac{\partial \Psi}{\partial \mathbf{u}_i}\right)^\top - \sum_{j=1}^{i-1} \left[ 
     \mathbf{u}^\top_j \left(\frac{\partial \mathbf{K}_j}{\partial \mathbf{u}_i}\right)^\top + \mathbf{C}^\top_{ji} + \sum^N_{k=j+1} \mathbf{u}^\top_k \left(\frac{\partial \mathbf{C}_{jk}}{\partial \mathbf{u}_i}\right)^\top \right]\Lambda_j  \quad \forall i \in [1, N] \,,
\end{equation}
thus yielding the final expression for the sensitivities
\begin{equation}
    \frac{\d \Psi^\prime}{\d \hat{\rho}}  = \frac{\partial \Psi}{\partial \hat{\rho}} + \sum^N_{i=1} \Lambda_{i}^{\top} \left( \frac{\partial \mathbf{K}_i}{\partial \hat{\rho}} \mathbf{u}_i - \frac{\partial \mathbf{f}_i}{\partial \hat{\rho}} + \sum^N_{j=i+1} \frac{\partial \mathbf{C}_{ij}}{\partial \hat{\rho}} \mathbf{u}_j \right)\,.
\end{equation}
This means that the adjoint equations in \eqref{eq:app_adj_eqs} are solved in reverse order with respect to the forward solution (\eqref{eq:app_multiphysics_weak}). In other words, one needs to solve the first adjoint equation ($i=1$) 
and feed the solution the next adjoint equation ($i=2$), and so on, to solve the last adjoint problem $i=N$  (as in the sequence: $\Lambda_1 \to \Lambda_{2} \to \cdots \to \Lambda_N$).

% In the sequential case, the solutions of the different physics couple one-to-one in a seggregated fashion (
% $\mathbf{u}_1 \to \mathbf{u}_2 \to \cdots \to \mathbf{u}_N$)
% and there is no other coupling mechanism between the physics. In this case the system of equations can be written as in Eq.~\eqref{eq:multiphysics_weak}.



% In this case, taking the derivative of the FOM with respect to the design variable $\xi$:
% \begin{equation}\label{eq:adj_seq}
%     \frac{\d \tilde{\Phi}}{\d \xi} = \frac{\partial \Phi}{\partial \xi} + \mathcal{D}^{(1)}_\circ \left[\Phi\right] + 
%     \sum^N_i \lambda_{i}^{\top} \left[ \left(\frac{\partial \mathbf{S}_i}{\partial \xi} +  \mathcal{D}^{(i)}_\circ \left[\mathbf{S}_i\right]\right) \mathbf{u}_i
%     + \mathbf{S}_i \mathcal{D}^{(i)}_\circ \left[\mathbf{u}_i\right] - \frac{\partial \mathbf{f}_i}{\partial \xi }\right]
% \end{equation}
% where we have defined $\mathcal{D}^{(i)}_\circ[a]$ as the sequential or composition ($\circ$) differential operator acting on $a$ to compactly write the derivative:
% \begin{align}
%     \mathcal{D}^{(i)}_\circ[a] &= \frac{\partial a}{\partial \mathbf{u}_i} \sum_{j=i}^{N} \frac{\partial \mathbf{u}_j}{\partial \xi} 
%         \prod_{k \leq j}^{j} \frac{\partial \mathbf{u}_k}{\partial \mathbf{u}_{k+1}}\,, \\
%         &= \frac{\partial a}{\partial \mathbf{u}_i} \left( \frac{\partial \mathbf{u}_i}{\partial \xi} + 
%             \frac{\partial \mathbf{u}_i}{\partial \mathbf{u}_{i+1}} \left( \frac{\partial \mathbf{u}_{i+1}}{\partial \xi} + 
%                 \frac{\partial \mathbf{u}_{i+1}}{\partial \mathbf{u}_{i+2}} \left( \frac{\partial \mathbf{u}_{i+2}}{\partial \xi} + 
%                     \cdots
%                 \right)
%             \right)
%         \right) \,.
%     \end{align}
% Using Eq.~\eqref{eq:adj_seq} and grouping the terms:
% \begin{equation}
%     \frac{\d \tilde{\Phi}}{\d \xi} =  \frac{\partial \Phi}{\partial \xi} + \sum_i \lambda_{i}^{\top} \left( \frac{\partial \mathbf{S}_i}{\partial \xi} \mathbf{u}_i - \frac{\partial \mathbf{f}_i}{\partial \xi} \right)
%     + \sum^{N-1}_{i=1}  \frac{\partial \mathbf{u}_{i}}{\partial \xi} \left( \lambda_{i}^{\top} \frac{\partial \mathbf{S}_i}{\partial \mathbf{u}_{i+1}}\mathbf{u}_{i}
%     +  \lambda_{i+1}^{\top} \mathbf{S}_{i+1}\right)\,.
% \end{equation}
% We can by choose the Lagrange multipliers to the last summation term is zero, by solving $N$ adjoint equations:
% \begin{align}
%     \mathbf{S}^\top_{1}\lambda_{1} &= - \frac{\partial \Phi}{\partial \mathbf{u}_{1}} \label{eq:seq_adj_1}\,\\
%     \mathbf{S}^\top_{i+1}\lambda_{i+1} &= - \mathbf{u}^\top_i \left(\frac{\partial \mathbf{S}_i}{\partial \mathbf{u}_{i+1}}\right)^\top \lambda_i \quad \forall i \in [1, N-1] \label{eq:seq_adj_N-1}\,,
% \end{align}
% where the adjoint equations imply a relationship where the coupling happens backwards. In other words, one needs to solve the first adjoint equation (Eq.~\eqref{eq:seq_adj_1}), 
% and feed the solution the next adjoint equation ($i=2$, Eq.~\eqref{eq:seq_adj_N-1}) (and so on); where the coupling is inverted with respect to the physics. Solving these adjoint equations we can calculate the lagrang
% multipliers which give the final sensitivities:
% \begin{equation}
%     \frac{\d \tilde{\Phi}}{\d \xi} = \frac{\partial \Phi}{\partial \xi} + \sum_i \lambda_{i}^{\top} \left( \frac{\partial \mathbf{S}_i}{\partial \xi} \mathbf{u}_i - \frac{\partial \mathbf{f}_i}{\partial \xi} \right)\,.
% \end{equation}

% \subsection{Parallel coupling}

% Let's now consider the case of parallel coupling, where the solution of $N-1$ physics are coupled solve the last ($i=1$) physics ($[\mathbf{u}_2, \mathbf{u}_3, \cdots, \mathbf{u}_{N-1}] \to \mathbf{u}_1$). By reusing Eq.~\eqref{eq:adj_seq},
% we can now take the derivative with respect to the design variables:
% \begin{align}\label{eq:adj_parallel}
%     \frac{\d \tilde{\Phi}}{\d \xi} &= \frac{\partial \Phi}{\partial \xi} + \mathcal{D}^{(1)}_\parallel \left[\Phi\right]  
%     +  \lambda_{1}^{\top} \left[\left( \frac{\partial \mathbf{S}_1}{\partial \xi} +  \mathcal{D}^{(1)}_\parallel \left[\mathbf{S}_1\right] \right) \mathbf{u}_1
%     + \mathbf{S}_1  \mathcal{D}^{(1)}_\parallel \left[\mathbf{u}_1\right] \right] + \\
%     &+ \sum_{i=2}^{N} \lambda_{i}^{\top} \left( \frac{\partial \mathbf{S}_i}{\partial \xi}\mathbf{u}_i + \mathbf{S}_i \frac{\partial \mathbf{u}_i}{\partial \xi} - \frac{\partial \mathbf{f}_i}{\partial \xi}\right)
% \end{align}
% where we have defined $\mathcal{D}^{(i)}_\parallel[a]$ as the parallel ($\parallel$) differential operator acting on $a$ to compactly write the derivative:
% \begin{equation}
%     D^{(i)}_\parallel[a] = \frac{\partial a}{\partial \mathbf{u}_i} \sum_{j=1}^{N} \frac{\partial \mathbf{u}_j}{\partial \mathbf{u}_i} 
%     \frac{\partial \mathbf{u}_j}{\partial \xi}\,.
% \end{equation}
% Using Eq.~\eqref{eq:adj_parallel} and grouping the terms:
% \begin{equation}
%     \frac{\d \tilde{\Phi}}{\d \xi} = \frac{\partial \Phi}{\partial \xi} + \sum_i \lambda_{i}^{\top} \left( \frac{\partial \mathbf{S}_i}{\partial \xi} \mathbf{u}_i - \frac{\partial \mathbf{f}_i}{\partial \xi} \right)
%     + \frac{\partial \mathbf{u}_1}{\partial \xi} \left( \lambda_{1}^{\top}  \mathbf{S}_1 + \frac{\partial \Phi}{\partial \mathbf{u}_1} \right) + 
%     \sum^{N}_{i=2} \left( \lambda_{i}^{\top} \mathbf{S}_i + \lambda_{1}^{\top}  \frac{\partial \mathbf{S}_i}{\partial \mathbf{u}_1} \mathbf{u}_i\right)\,.
% \end{equation}
% We can by choose the Lagrange multipliers so that the last two terms are zero, by solving $N$ adjoint equations:
% \begin{align}
%     \mathbf{S}^\top_{1}\lambda_{1} &= - \frac{\partial \Phi}{\partial \mathbf{u}_{1}} \label{eq:par_adj_1}\,\\
%     \mathbf{S}^\top_{i}\lambda_{i} &= - \mathbf{u}^\top_1 \left(\frac{\partial \mathbf{S}_1}{\partial \mathbf{u}_i}\right)^\top \lambda_1 \quad \forall i \in [2, N] \label{eq:par_adj_N-1}\,,
% \end{align}
% The Lagrange multipliers are a solution to this equation and can be used to simplify the sensitivity expression:
% \begin{equation}
%     \frac{\d \tilde{\Phi}}{\d \xi} = \frac{\partial \Phi}{\partial \xi} + \sum_i \lambda_{i}^{\top} \left( \frac{\partial \mathbf{S}_i}{\partial \xi} \mathbf{u}_i - \frac{\partial \mathbf{f}_i}{\partial \xi} \right)\,.
% \end{equation}
% which is the same result as in the case of sequential coupling, with different Lagrange multipliers.

% \subsection{Generalizing to simultaenous sequential and parallel coupling}
% Based on the result from the previous sections, in the general coupling case, where there might be parallel and sequential couplings simultaneously, the sensitivities can be calculated as:
% \begin{equation}
%     \frac{\d \tilde{\Phi}}{\d \xi} = \frac{\partial \Phi}{\partial \xi} + \sum_i \lambda_{i}^{\top} \left( \frac{\partial \mathbf{S}_i}{\partial \xi} \mathbf{u}_i - \frac{\partial \mathbf{f}_i}{\partial \xi} \right)\,.
% \end{equation}
% The only difference is solution to the adjoint equations, which depends on the coupling. In the general case, where all couplings feed to a 
% final physics ($i=1$) but may be coupled in any combinations between each other, it can be shown that the adjoint equations are:
% \begin{align}
%     \mathbf{S}^\top_{1}\lambda_{1} &= - \frac{\partial \Phi}{\partial \mathbf{u}_{1}}\,\\
%     \mathbf{S}^\top_{i}\lambda_{i} &= - \sum^{C_i}_j \mathbf{u}^\top_j \left(\frac{\partial \mathbf{S}_j}{\partial \mathbf{u}_{i}}\right)^\top \lambda_j \quad \forall i \in [1, N-1]\, , \quad \forall j \in [1, C_i] \,.
% \end{align}
% where $C_i<N$ is the number of physics that are coupled to the $i$th physics, and where we have used that the problems are linear. 


