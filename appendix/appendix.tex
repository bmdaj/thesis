\section{Adjoint sensitivity analysis in coupled mutliphysics problems -- sequential and parallel coupling}\label{app:appendix1}

In this appendix we derive the expressions for the adjoint sensitivity analysis in seggregatedly coupled multiphysics problems.
 This means that the coupling is unidirectinal, so that $\mathbf{u}_1 \to \mathbf{u}_2$ and $\mathbf{u}_1 \not\to \mathbf{u}_2$, for two
 coupled state fields $\mathbf{u}_1$ and $\mathbf{u}_2$. For simplicity, we consider
only real fields. We also restrict ourselves to coupling through material properties (Sec. XX), but this can be extended to other couplings
(e.g., source terms, Sec. YY).\\

In the most general case of $N$ coupled physics problems we can rewrite the FOM by adding the residual
of the state equations, multiplied by the Lagrange multipliers, $\lambda_i$, 
respectively:
\begin{equation}\label{eq:adj_init}
    \tilde{\Phi} =\Phi + \sum^N_i \lambda_{i}^{\top}\left(\mathbf{S}_i \mathbf{u}_i -\mathbf{f}_i\right)\,.
\end{equation}
Now we will see how this can be used to calculate the sensitivities when the physics are coupled in a sequential or parallel way.

\subsection{Sequential coupling}

In the sequential case, the solutions of the different physics couple one-to-one in a seggregated fashion (
$\mathbf{u}_1 \to \mathbf{u}_2 \to \cdots \to \mathbf{u}_N$)
and there is no other coupling mechanism between the physics. In this case, taking the derivative of the FOM with respect to the design variable $\xi$:
\begin{equation}\label{eq:adj_seq}
    \frac{\d \tilde{\Phi}}{\d \xi} = \frac{\partial \Phi}{\partial \xi} + \mathcal{D}^{(1)}_\circ \left[\Phi\right] + 
    \sum^N_i \lambda_{i}^{\top} \left[ \left(\frac{\partial \mathbf{S}_i}{\partial \xi} +  \mathcal{D}^{(i)}_\circ \left[\mathbf{S}_i\right]\right) \mathbf{u}_i
    + \mathbf{S}_i \mathcal{D}^{(i)}_\circ \left[\mathbf{u}_i\right] - \frac{\partial \mathbf{f}_i}{\partial \xi }\right]
\end{equation}
where we have defined $\mathcal{D}^{(i)}_\circ[a]$ as the sequential or composition ($\circ$) differential operator acting on $a$ to compactly write the derivative:
\begin{align}
    \mathcal{D}^{(i)}_\circ[a] &= \frac{\partial a}{\partial \mathbf{u}_i} \sum_{j=i}^{N} \frac{\partial \mathbf{u}_j}{\partial \xi} 
        \prod_{k \leq j}^{j} \frac{\partial \mathbf{u}_k}{\partial \mathbf{u}_{k+1}}\,, \\
        &= \frac{\partial a}{\partial \mathbf{u}_i} \left( \frac{\partial \mathbf{u}_i}{\partial \xi} + 
            \frac{\partial \mathbf{u}_i}{\partial \mathbf{u}_{i+1}} \left( \frac{\partial \mathbf{u}_{i+1}}{\partial \xi} + 
                \frac{\partial \mathbf{u}_{i+1}}{\partial \mathbf{u}_{i+2}} \left( \frac{\partial \mathbf{u}_{i+2}}{\partial \xi} + 
                    \cdots
                \right)
            \right)
        \right) \,.
    \end{align}
Using Eq.~\eqref{eq:adj_seq} and grouping the terms:
\begin{equation}
    \frac{\d \tilde{\Phi}}{\d \xi} =  \frac{\partial \Phi}{\partial \xi} + \sum_i \lambda_{i}^{\top} \left( \frac{\partial \mathbf{S}_i}{\partial \xi} \mathbf{u}_i - \frac{\partial \mathbf{f}_i}{\partial \xi} \right)
    + \sum^{N-1}_{i=1}  \frac{\partial \mathbf{u}_{i}}{\partial \xi} \left( \lambda_{i}^{\top} \frac{\partial \mathbf{S}_i}{\partial \mathbf{u}_{i+1}}\mathbf{u}_{i}
    +  \lambda_{i+1}^{\top} \mathbf{S}_{i+1}\right)\,.
\end{equation}
We can by choose the Lagrange multipliers to the last summation term is zero, by solving $N$ adjoint equations:
\begin{align}
    \mathbf{S}^\top_{1}\lambda_{1} &= - \frac{\partial \Phi}{\partial \mathbf{u}_{1}} \label{eq:seq_adj_1}\,\\
    \mathbf{S}^\top_{i+1}\lambda_{i+1} &= - \mathbf{u}^\top_i \left(\frac{\partial \mathbf{S}_i}{\partial \mathbf{u}_{i+1}}\right)^\top \lambda_i \quad \forall i \in [1, N-1] \label{eq:seq_adj_N-1}\,,
\end{align}
where the adjoint equations imply a relationship where the coupling happens backwards. In other words, one needs to solve the first adjoint equation (Eq.~\eqref{eq:seq_adj_1}), 
and feed the solution the next adjoint equation ($i=2$, Eq.~\eqref{eq:seq_adj_N-1}) (and so on); where the coupling is inverted with respect to the physics. Solving these adjoint equations we can calculate the lagrang
multipliers which give the final sensitivities:
\begin{equation}
    \frac{\d \tilde{\Phi}}{\d \xi} = \frac{\partial \Phi}{\partial \xi} + \sum_i \lambda_{i}^{\top} \left( \frac{\partial \mathbf{S}_i}{\partial \xi} \mathbf{u}_i - \frac{\partial \mathbf{f}_i}{\partial \xi} \right)\,.
\end{equation}

\subsection{Parallel coupling}

Let's now consider the case of parallel coupling, where the solution of $N-1$ physics are coupled solve the last ($i=1$) physics ($[\mathbf{u}_2, \mathbf{u}_3, \cdots, \mathbf{u}_{N-1}] \to \mathbf{u}_1$). By reusing Eq.~\eqref{eq:adj_seq},
we can now take the derivative with respect to the design variables:
\begin{align}\label{eq:adj_parallel}
    \frac{\d \tilde{\Phi}}{\d \xi} &= \frac{\partial \Phi}{\partial \xi} + \mathcal{D}^{(1)}_\parallel \left[\Phi\right]  
    +  \lambda_{1}^{\top} \left[\left( \frac{\partial \mathbf{S}_1}{\partial \xi} +  \mathcal{D}^{(1)}_\parallel \left[\mathbf{S}_1\right] \right) \mathbf{u}_1
    + \mathbf{S}_1  \mathcal{D}^{(1)}_\parallel \left[\mathbf{u}_1\right] \right] + \\
    &+ \sum_{i=2}^{N} \lambda_{i}^{\top} \left( \frac{\partial \mathbf{S}_i}{\partial \xi}\mathbf{u}_i + \mathbf{S}_i \frac{\partial \mathbf{u}_i}{\partial \xi} - \frac{\partial \mathbf{f}_i}{\partial \xi}\right)
\end{align}
where we have defined $\mathcal{D}^{(i)}_\parallel[a]$ as the parallel ($\parallel$) differential operator acting on $a$ to compactly write the derivative:
\begin{equation}
    D^{(i)}_\parallel[a] = \frac{\partial a}{\partial \mathbf{u}_i} \sum_{j=1}^{N} \frac{\partial \mathbf{u}_j}{\partial \mathbf{u}_i} 
    \frac{\partial \mathbf{u}_j}{\partial \xi}\,.
\end{equation}
Using Eq.~\eqref{eq:adj_parallel} and grouping the terms:
\begin{equation}
    \frac{\d \tilde{\Phi}}{\d \xi} = \frac{\partial \Phi}{\partial \xi} + \sum_i \lambda_{i}^{\top} \left( \frac{\partial \mathbf{S}_i}{\partial \xi} \mathbf{u}_i - \frac{\partial \mathbf{f}_i}{\partial \xi} \right)
    + \frac{\partial \mathbf{u}_1}{\partial \xi} \left( \lambda_{1}^{\top}  \mathbf{S}_1 + \frac{\partial \Phi}{\partial \mathbf{u}_1} \right) + 
    \sum^{N}_{i=2} \left( \lambda_{i}^{\top} \mathbf{S}_i + \lambda_{1}^{\top}  \frac{\partial \mathbf{S}_i}{\partial \mathbf{u}_1} \mathbf{u}_i\right)\,.
\end{equation}
We can by choose the Lagrange multipliers so that the last two terms are zero, by solving $N$ adjoint equations:
\begin{align}
    \mathbf{S}^\top_{1}\lambda_{1} &= - \frac{\partial \Phi}{\partial \mathbf{u}_{1}} \label{eq:par_adj_1}\,\\
    \mathbf{S}^\top_{i}\lambda_{i} &= - \mathbf{u}^\top_1 \left(\frac{\partial \mathbf{S}_1}{\partial \mathbf{u}_i}\right)^\top \lambda_1 \quad \forall i \in [2, N] \label{eq:par_adj_N-1}\,,
\end{align}
The Lagrange multipliers are a solution to this equation and can be used to simplify the sensitivity expression:
\begin{equation}
    \frac{\d \tilde{\Phi}}{\d \xi} = \frac{\partial \Phi}{\partial \xi} + \sum_i \lambda_{i}^{\top} \left( \frac{\partial \mathbf{S}_i}{\partial \xi} \mathbf{u}_i - \frac{\partial \mathbf{f}_i}{\partial \xi} \right)\,.
\end{equation}
which is the same result as in the case of sequential coupling, with different Lagrange multipliers.

\subsection{Generalizing to simultaenous sequential and parallel coupling}
Based on the result from the previous sections, in the general coupling case, where there might be parallel and sequential couplings simultaneously, the sensitivities can be calculated as:
\begin{equation}
    \frac{\d \tilde{\Phi}}{\d \xi} = \frac{\partial \Phi}{\partial \xi} + \sum_i \lambda_{i}^{\top} \left( \frac{\partial \mathbf{S}_i}{\partial \xi} \mathbf{u}_i - \frac{\partial \mathbf{f}_i}{\partial \xi} \right)\,.
\end{equation}
The only difference is solution to the adjoint equations, which depends on the coupling. In the general case, where all couplings feed to a 
final physics ($i=1$) but may be coupled in any combinations between each other, it can be shown that the adjoint equations are:
\begin{align}
    \mathbf{S}^\top_{1}\lambda_{1} &= - \frac{\partial \Phi}{\partial \mathbf{u}_{1}}\,\\
    \mathbf{S}^\top_{i}\lambda_{i} &= - \sum^{C_i}_j \mathbf{u}^\top_j \left(\frac{\partial \mathbf{S}_j}{\partial \mathbf{u}_{i}}\right)^\top \lambda_j \quad \forall i \in [1, N-1]\, , \quad \forall j \in [1, C_i] \,.
\end{align}
where $C_i<N$ is the number of physics that are coupled to the $i$th physics, and where we have used that the problems are linear. 
Note that the couplings are still unidirectional. To extend this derivations to complex fields one can refer to the adjoint sensitivity
analysis in \cite{ownpub0}, for a simpler two-physics ($N=2$) problem, where the optical problem uses complex fields.

