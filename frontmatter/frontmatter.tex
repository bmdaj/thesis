\begin{titlingpage}
  \titleM
  \clearpage
  \noindent\textbf{Thesis title:} \\\noindent Multiphysics topology optimization in nanophotonics
  \vspace{1.0em}

  \noindent\textbf{Author:}\\\noindent Beñat Martinez de Aguirre Jokisch\\Department of Civil and Mechanical Engineering\\ Technical University of Denmark
  \vspace{0.5em}

  \noindent\textbf{Main Supervisor:}\\\noindent Ole Sigmund\\Department of Civil and Mechanical Engineering\\ Technical University of Denmark
  \vspace{0.5em}

  \noindent\textbf{Co-supervisors:}\\
  \noindent Rasmus Ellebæk Christiansen\\Department of Civil and Mechanical Engineering\\ Technical University of Denmark
  \vspace{0.25em}

  \noindent Jesper Mørk\\Department of Electrical and Photonics Engineering\\ Technical University of Denmark
  \vspace{0.5em}

  \noindent\textbf{Funding and competing interests:}\\\noindent This work was funded by the NanoPhoton Research Center, through the Danish National Research Foundation (DNRF147). The author has no competing interests.
  \vspace{1.0em}

  \noindent\textbf{Duration:}\\\noindent The work has been carried out between the $1^\text{st}$ of August 2022 and the $31^\text{st}$ of July 2025.
  \vspace{1.0em}

  \noindent \copyright Beñat Martinez de Aguirre Jokisch
  \vspace{1em}

  \noindent Department of Civil and Mechanical Engineering\\Technical University of Denmark\\ Nils Koppels Allé, Building 404 \\ DK-2800 Kongens Lyngby, Denmark
  \vspace{1em}

  \noindent MEK-PHD ISSN: 0903-1685
\end{titlingpage}
\setcounter{page}{3}
\chapter*{Preface}
This thesis is submitted in partial fulfillment of the requirements for obtaining
the PhD degree at the Technical University of Denmark
(DTU). The Danish National Research Foundation has funded the project as a part of
the NanoPhoton -- Center for Nanophotonics. The work has been carried out at the
Section for Solid Mechanics at the Department of Civil and Mechanical Engineering
at DTU. The project has been carried out between the 1$^\text{st}$ of August 2022 and the 31$^\text{st}$
of July 2025. The main supervisor has been Professor Ole Sigmund with co-supervisors Associate Professor Rasmus Ellebæk Christiansen and Professor Jesper
Mørk.

First and foremost, I would like to thank my supervisors. Ole, thank you for teaching me the \emph{art} of topology optimization, for your guidance, and for always
finding some time to explore research ideas. Rasmus, thank you for all your help and for always having the office door open when I needed to discuss.
Jesper, thank you for always supporting my endeavors and for your insights on laser physics. I also want to thank Steven G. Johnson for welcoming me to his group at MIT and teaching me so many things about photonics,
mathematics, music and US culture. A special thanks to my friend Rodrigo for all the good times outside of MIT and the fun chess sessions 
during the breaks at work.

I would also like to thank my colleagues. Benjamin, thank you
for a productive research collaboration and the fun times in California. To my office mates and friends, Jonathan, Hossein, Göktuğ, Asger and Jens, thank you for creating 
a super fun and, \emph{at times}, even productive work environment. And of course, thanks for surviving my shoot-hard, think-later \emph{basque} fussball style.


Finally, I want to thank my family and friends for their invaluable support. Thank you, Alba, for always being there. 
And to my parents, Berni and Katja, and my sister Nahia, thank you for always believing in me.

\noindent Kongens Lyngby, \today,\\
\vspace{0.1cm}\\
\noindent \textit{Beñat Martinez de Aguirre Jokisch}

\chapter*{Abstract}

This Ph.D. thesis addresses the development of multiphysics topology optimization methods and their application to nanophotonic device design. The work covers three multiphysics systems: thermo-optical, optomechanical, and electro-optical systems.

In the thermo-optical domain, the design of phase shifters for integrated photonic circuits is investigated. Using topology optimization, the distribution of a resistive metallic heater around an optical waveguide is engineered to achieve a low optical loss phase shift between heated and unheated states. This approach reduces optical losses by $\approx 33\%$ compared to state-of-the-art designs.

In the optomechanical domain, several design problems are explored. First, particle and environment geometries are optimized to enhance optical forces, resulting in up to a $\approx 13 \times$ improvement over a reference design. Compact on-chip optical cavities capable of omnidirectional trapping of sub-wavelength particles are then designed, improving the trapping stiffness by an order of magnitude compared to conventional optical tweezers. Finally, the design of an optomechanical membrane device is optimized, showing that incorporating optical-force-induced deformation into the model yields a $\approx 3 \times$ performance increase over designs that neglect this strong coupling effect.

In the electro-optical domain, nanolaser design is studied. An efficient figure of merit that incorporates the underlying laser physics, including carrier diffusion, is derived. This formulation leads to up to a $\approx 3\times$ performance improvement over traditional optimization approaches that do not account for these effects.

\chapter*{Resumé}
Denne ph.d.-afhandling omhandler udviklingen af mul\-ti\-fy\-siske to\-po\-lo\-gi\-op\-ti\-merings\-me\-to\-der og deres anvendelse til design af nanofotoniske komponenter. Arbejdet dækker tre multifysiske systemer: termo-optiske, opto\-mekaniske og elektro-optiske systemer.

Inden for det termo-optiske område undersøges designet af fase\-skiftere til integrerede fotoniske kredsløb. Ved hjælp af topologioptimering designes fordelingen af et elektrisk varmelegeme omkring en optisk bølgeleder, hvil\-ket muliggør et faseskift med lavt optisk tab mellem den opvarmede og den ikke-opvarmede tilstand. Denne tilgang reducerer de optiske tab med $\approx 33\%$ sammenlignet med state-of-the-art løsninger.

Inden for det optomekaniske område behandles flere designproblemer. Først optimeres partikel- og omgivelsesgeometrier for at forstærke optiske kræfter, hvilket giver op til en $\approx 13 \times$ forbedring i forhold til en referencestruktur. Dernæst designes kompakte on-chip optiske kaviteter, der kan indfange sub-bølgelængde partikler i alle retninger, hvilket giver en væsentligt stærkere fastholdelse af partiklerne sammenlignet med konventionelle optiske pincetter. Endelig optimeres designet af en optomekanisk membran\-struktur, og det vises, at inklusion af optisk kraft-induceret deformation i modellen giver en $\approx 3 \times$ bedre ydeevne sammenlignet med design, der ikke inkluderer denne stærke koblingseffekt.

Inden for det elektro-optiske område undersøges designet af nanolasere. En effektiv objektfunktion, der inkluderer den underliggende laser\-fy\-sik, herunder ladningsbærerdiffusion, udledes. Denne formulering medfører op til en \mbox{$\approx 3 \times$} forbe\-dring af ydeevnen i forhold til traditionelle optimeringsmetoder, der ikke tager højde for disse effekter.

\cleardoublepage
\chapter*{Declaration of generative-AI assistance}

The author applied the large-language-model ChatGPT (o3, o4-mini) in a supportive role
during the writing of this thesis. The model's output was employed for:
\begin{itemize}
\item assisting with high-level proofreading of the outline and logical structure;
\item providing suggestions for the Danish translation of the abstract (\emph{resumé});
\item suggesting alternative phrasings to improve clarity, style, and overall flow; and
\item generating code snippets to enhance computational speed through vectorized operations and just-in-time compilation.
\end{itemize}
All suggestions were critically reviewed, and where necessary, rewritten by the author before
inclusion. Responsibility for the final text, code, and scientifc conclusions rests with the author.

\cleardoublepage
\chapter*{List of Publications}
\nocite{ownpub0,ownpub1,ownpub3, ownpub2, ownpub4, ownpub5}
\newrefcontext[sorting=none,labelprefix=P]
\printbibliography[env=bibliographyNUM,keyword=myPub,title={List of publications},heading=none,resetnumbers]
\newrefcontext[sorting=none,labelprefix=M]
\printbibliography[env=bibliographyNUM,keyword=myMan,heading=none]
%https://tex.stackexchange.com/questions/553753/how-to-add-list-of-publication-in-thesis-class
%\endrefcontext

\cleardoublepage

{
  \hypersetup{linkcolor=black}
  \tableofcontents*
}


