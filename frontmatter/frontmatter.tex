\begin{titlingpage}
  \title
  \clearpage
  \noindent\textbf{Thesis title:} \\\noindent Multiphysics topology optimization in nanophotonics
  \vspace{1.5em}

  \noindent\textbf{Author:}\\\noindent Beñat Martinez de Aguirre Jokisch\\Department of Civil and Mechanical Engineering\\ Technical University of Denmark
  \vspace{0.5em}

  \noindent\textbf{Main Supervisor:}\\\noindent Ole Sigmund\\Department of Civil and Mechanical Engineering\\ Technical University of Denmark
  \vspace{0.5em}

  \noindent\textbf{Co-supervisors:}\\
  \noindent Rasmus Ellebæk Christiansen\\Department of Civil and Mechanical Engineering\\ Technical University of Denmark
  \vspace{0.5em}

  \noindent Jesper Mørk\\Department of Electrical and Photonics Engineering\\ Technical University of Denmark
  \vspace{0.5em}

  \noindent\textbf{Funding and competing interests:}\\\noindent This work was funded by the NanoPhoton Research Center, through the Danish National Research Foundation (DNRF147). The author has no competing interests.
  \vspace{1.5em}

  \noindent\textbf{Duration:}\\\noindent The work has been carried out between the August 2022 and the July 2025.
  \vspace{2.2em}

  \noindent \copyright Beñat Martinez de Aguirre Jokisch
  \vspace{1em}

  \noindent Department of Civil and Mechanical Engineering\\Technical University of Denmark\\ Nils Koppels Allé, Building 404 \\ DK-2800 Kongens Lyngby, Denmark
  \vspace{1em}

  \noindent MEK-PHD ISSN: \#\#\#\#-\#\#\#\#
\end{titlingpage}
\setcounter{page}{3}
\chapter*{Preface}
This thesis is submitted in partial fulfillment of the requirements for obtaining
the PhD degree at the Technical University of Denmark
(DTU). The Danish National Research Foundation funded the project as a part of
the NanoPhoton -- Center for Nanophotonics. The work has been carried out at the
Section for Solid Mechanics at the Department of Civil and Mechanical Engineering
at DTU. The project was carried out between the 1$^\text{st}$ of August 2022 and the 31$^\text{st}$
of July 2025. The main supervisor has been Professor Ole Sigmund with cosupervisors Associate Professor Rasmus Ellebæk Christiansen and Professor Jesper
Mørk.

First and foremost, I would like to thank my supervisors. Ole, thank you for teaching me the \emph{art} of topology optimization, for your guidance, and for always
finding some time to discuss research ideas. Rasmus, thank you for all your support and your insights on numerical implementation.
Jesper, thank you for always supporting my endeavors and for discussing 
laser physics with me. I also want to thank Steven G. Johnson for welcoming me to his group at MIT and teaching me so many things about photonics,
mathematics, music and US culture. A special thanks to my friend Rodrigo for all the fun activities outside of MIT and the fun chess games 
during the breaks.

I would also like to thank my colleagues, especially my office mates and friends, Jonathan, Hossein, Göktuğ, Asger and Jens, for creating 
a super fun and \emph{sometimes} productive work environment. Guys, thanks for helping me improve my fussball skills.


Finally, I would like to thank my family and friends for their invaluable support. Thank you Alba for always being there with me.
Thanks to my parents, Berni and Katja, and my sister Nahia for always believing in me.

\noindent Kongens Lyngby, \today,\\
\vspace{0.1cm}\\
\noindent \textit{Beñat Martinez de Aguirre Jokisch}

\chapter*{Abstract}

Thus Ph.D. thesis addresses the development of multiphysics topology optimization methods and their application to 
nanophotonic device design. The work mainly focuses on three multiphysics problems: thermo-optical, optomechanical, 
and electro-optic problems.


\chapter*{Resumé}

Bla bla bla...

\chapter*{Declaration of generative-AI assistance}

The author applied the large-language-model ChatGPT (o3, o4-mini) in a supportive role
during the writing of this thesis. The model's output was employed for:
\begin{itemize}
\item high-level proofreading of the report outline and section flow;
\item suggesting alternative phrasings to improve clarity and style; and
\item generating code fragments that were used for computational speed-up and figure generation.
\end{itemize}
All suggestions were critically reviewed, and where necessary, rewritten by the author before
inclusion. Responsibility for the final text, code, and scientifc conclusions rests with the author.

\cleardoublepage
\chapter*{List of Publications}
\nocite{ownpub0,ownpub1,ownpub3, ownpub2, ownpub4}
\newrefcontext[sorting=none,labelprefix=P]
\printbibliography[env=bibliographyNUM,keyword=myPub,title={List of publications},heading=none,resetnumbers]
\newrefcontext[sorting=none,labelprefix=M]
\printbibliography[env=bibliographyNUM,keyword=myMan,heading=none]
%https://tex.stackexchange.com/questions/553753/how-to-add-list-of-publication-in-thesis-class
%\endrefcontext

\cleardoublepage

{
  \hypersetup{linkcolor=black}
  \tableofcontents*
}

\chapter*{}

\begin{flushright}
  \textit{Nire adiskide Mikeli,}
\end{flushright}

