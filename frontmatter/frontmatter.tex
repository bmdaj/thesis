\begin{titlingpage}
  \titleM
  \clearpage
  \noindent\textbf{Thesis title:} \\\noindent Multiphysics topology optimization in nanophotonics
  \vspace{1.0em}

  \noindent\textbf{Author:}\\\noindent Beñat Martinez de Aguirre Jokisch\\Department of Civil and Mechanical Engineering\\ Technical University of Denmark
  \vspace{0.5em}

  \noindent\textbf{Main Supervisor:}\\\noindent Ole Sigmund\\Department of Civil and Mechanical Engineering\\ Technical University of Denmark
  \vspace{0.5em}

  \noindent\textbf{Co-supervisors:}\\
  \noindent Rasmus Ellebæk Christiansen\\Department of Civil and Mechanical Engineering\\ Technical University of Denmark
  \vspace{0.25em}

  \noindent Jesper Mørk\\Department of Electrical and Photonics Engineering\\ Technical University of Denmark
  \vspace{0.5em}

  \noindent\textbf{Funding and competing interests:}\\\noindent This work was funded by the NanoPhoton Research Center, through the Danish National Research Foundation (DNRF147). The author has no competing interests.
  \vspace{1.0em}

  \noindent\textbf{Duration:}\\\noindent The work has been carried out between the $1^\text{st}$ of August 2022 and the $31^\text{st}$ of July 2025.
  \vspace{1.0em}

  \noindent \copyright Beñat Martinez de Aguirre Jokisch
  \vspace{1em}

  \noindent Department of Civil and Mechanical Engineering\\Technical University of Denmark\\ Nils Koppels Allé, Building 404 \\ DK-2800 Kongens Lyngby, Denmark
  \vspace{1em}

  \noindent MEK-PHD ISSN: 0903-1685
\end{titlingpage}
\setcounter{page}{3}
\chapter*{Preface}
This thesis is submitted in partial fulfillment of the requirements for obtaining
the PhD degree at the Technical University of Denmark
(DTU). The Danish National Research Foundation funded the project as a part of
the NanoPhoton -- Center for Nanophotonics. The work has been carried out at the
Section for Solid Mechanics at the Department of Civil and Mechanical Engineering
at DTU. The project was carried out between the 1$^\text{st}$ of August 2022 and the 31$^\text{st}$
of July 2025. The main supervisor has been Professor Ole Sigmund with cosupervisors Associate Professor Rasmus Ellebæk Christiansen and Professor Jesper
Mørk.

First and foremost, I would like to thank my supervisors. Ole, thank you for teaching me the \emph{art} of topology optimization, for your guidance, and for always
finding some time to discuss research ideas. Rasmus, thank you for all your help and for always having office door open when I needed to discuss.
Jesper, thank you for always supporting my endeavors and for discussing 
laser physics with me. I also want to thank Steven G. Johnson for welcoming me to his group at MIT and teaching me so many things about photonics,
mathematics, music and US culture. A special thanks to my friend Rodrigo for all the fun activities outside of MIT and the fun chess games 
during the breaks.

I would also like to thank my colleagues. Benjamin, thank you
for a productive research collaboration and the fun times in California. To my office mates and friends, Jonathan, Hossein, Göktuğ, Asger and Jens, thank you for creating 
a super fun and, \emph{at times}, even productive work environment. And of course, thanks for surviving my shoot-hard, think-later \emph{basque} fussball style.


Finally, I want to thank my family and friends for their invaluable support. Thank you, Alba, for always being there. 
And to my parents, Berni and Katja, and my sister Nahia, thank you for always believing in me.

\noindent Kongens Lyngby, \today,\\
\vspace{0.1cm}\\
\noindent \textit{Beñat Martinez de Aguirre Jokisch}

\chapter*{Abstract}

This Ph.D. thesis focuses on the development of multiphysics topology optimization methods and their application to nanophotonic device design. The work focuses on three multiphysics systems: thermo-optical, optomechanical, and electro-optical systems.

In thermo-optical domain, we focus on the design of phase shifters for integrated photonic circuits. Using topology optimization, we engineer the distribution of a resistive metallic heater around an optical waveguide to achieve a low optical loss phase shift between heated and unheated states. This approach reduces optical losses by $33\%$ compared to state-of-the-art designs.

In the optomechanical domain we explore several design problems. First, we optimize particle and environment geometries to enhance optical forces, achieving up to a $\approx 13 \times$ improvement over a reference design. We then design compact on-chip optical cavities capable of omnidirectional trapping of sub-wavelength particles, improving trapping stiffness by an order of magnitude compared to conventional optical tweezers. Finally, we optimize the design of an optomechanical membrane device, showing that incorporating optical-force-induced deformation into the model yields a $\approx 3 \times$ performance increase over designs that neglect this strong coupling effect.

In the electro-optical domain, we address nanolaser design. We derive an efficient figure of merit that incorporates the underlying laser physics, including carrier diffusion. This formulation leads to a $\approx 3\times$ performance improvement over traditional optimization approaches that do not account for these effects.

\chapter*{Resumé}

Denne Ph.d.-afhandling omhandler udviklingen af multiphysik topologioptimeringsmetoder og deres anvendelse til design af nanofotoniske komponenter. Arbejdet fokuserer på tre centrale multiphysiksystemer: termo-optiske, optomekaniske og elektro-optiske systemer.

Inden for det termo-optiske område fokuseres der på design af faseskiftere til integrerede fotoniske kredsløb. Ved hjælp af topologioptimering designes fordelingen af en resistiv metallisk varmelegeme omkring en optisk bølgeleder, hvilket muliggør et faseskift med lavt optisk tab mellem opvarmede og ikke-opvarmede tilstande. Denne metode reducerer de optiske tab med cirka $33\%$ sammenlignet med nogle af de bedste eksisterende løsninger.

Inden for det optomekaniske område undersøges flere designproblemer. Først optimeres partikler og deres omgivelser for at forstærke optiske kræfter, hvilket giver op til $\approx 13 \times$ forbedring i forhold til en referencestruktur. Dernæst designes kompakte optiske hulrum på chippen, som kan fange sub-bølgelængde partikler i alle retninger, og forbedrer fangststivheden med en størrelsesorden i forhold til konventionelle optiske pincetter. Endelig optimeres en optomekanisk membranstruktur, hvor modellering af deformation forårsaget af optiske kræfter giver en $\approx 3 \times$ forbedring i ydeevne sammenlignet med modeller, der ikke inkluderer denne stærke kobling.

Inden for det elektro-optiske område behandles design af nanolasere. Her udvikles en effektiv målfunktion, som inkluderer den underliggende laserfysik, herunder ladningsbærer-diffusion. Denne formulering giver en $\approx 3\times$ forbedring i ydeevne sammenlignet med traditionelle optimeringsmetoder, der ikke tager højde for disse fysiske effekter.

\cleardoublepage
\chapter*{Declaration of generative-AI assistance}

The author applied the large-language-model ChatGPT (o3, o4-mini) in a supportive role
during the writing of this thesis. The model's output was employed for:
\begin{itemize}
\item high-level proofreading of the report outline and section flow;
\item suggesting alternative phrasings to improve clarity and style; and
\item generating code fragments that were used for computational speed-up and figure generation.
\end{itemize}
All suggestions were critically reviewed, and where necessary, rewritten by the author before
inclusion. Responsibility for the final text, code, and scientifc conclusions rests with the author.

\cleardoublepage
\chapter*{List of Publications}
\nocite{ownpub0,ownpub1,ownpub3, ownpub2, ownpub4, ownpub5}
\newrefcontext[sorting=none,labelprefix=P]
\printbibliography[env=bibliographyNUM,keyword=myPub,title={List of publications},heading=none,resetnumbers]
\newrefcontext[sorting=none,labelprefix=M]
\printbibliography[env=bibliographyNUM,keyword=myMan,heading=none]
%https://tex.stackexchange.com/questions/553753/how-to-add-list-of-publication-in-thesis-class
%\endrefcontext

\cleardoublepage

{
  \hypersetup{linkcolor=black}
  \tableofcontents*
}


